% Options for packages loaded elsewhere
\PassOptionsToPackage{unicode}{hyperref}
\PassOptionsToPackage{hyphens}{url}
%
\documentclass[
  english,
  man,floatsintext]{apa6}
\title{Measuring individual differences in the understanding of gaze cues across the lifespan}
\author{Julia Prein\textsuperscript{1}, Manuel Bohn\textsuperscript{1}, Luke Maurits\textsuperscript{1}, Steven Kalinke\textsuperscript{1}, \& Daniel M. Haun\textsuperscript{1}}
\date{}

\usepackage{amsmath,amssymb}
\usepackage{lmodern}
\usepackage{iftex}
\ifPDFTeX
  \usepackage[T1]{fontenc}
  \usepackage[utf8]{inputenc}
  \usepackage{textcomp} % provide euro and other symbols
\else % if luatex or xetex
  \usepackage{unicode-math}
  \defaultfontfeatures{Scale=MatchLowercase}
  \defaultfontfeatures[\rmfamily]{Ligatures=TeX,Scale=1}
\fi
% Use upquote if available, for straight quotes in verbatim environments
\IfFileExists{upquote.sty}{\usepackage{upquote}}{}
\IfFileExists{microtype.sty}{% use microtype if available
  \usepackage[]{microtype}
  \UseMicrotypeSet[protrusion]{basicmath} % disable protrusion for tt fonts
}{}
\makeatletter
\@ifundefined{KOMAClassName}{% if non-KOMA class
  \IfFileExists{parskip.sty}{%
    \usepackage{parskip}
  }{% else
    \setlength{\parindent}{0pt}
    \setlength{\parskip}{6pt plus 2pt minus 1pt}}
}{% if KOMA class
  \KOMAoptions{parskip=half}}
\makeatother
\usepackage{xcolor}
\IfFileExists{xurl.sty}{\usepackage{xurl}}{} % add URL line breaks if available
\IfFileExists{bookmark.sty}{\usepackage{bookmark}}{\usepackage{hyperref}}
\hypersetup{
  pdftitle={Measuring individual differences in the understanding of gaze cues across the lifespan},
  pdfauthor={Julia Prein1, Manuel Bohn1, Luke Maurits1, Steven Kalinke1, \& Daniel M. Haun1},
  pdflang={en-EN},
  pdfkeywords={social cognition, individual differences, gaze cues, psychometrics},
  hidelinks,
  pdfcreator={LaTeX via pandoc}}
\urlstyle{same} % disable monospaced font for URLs
\usepackage{graphicx}
\makeatletter
\def\maxwidth{\ifdim\Gin@nat@width>\linewidth\linewidth\else\Gin@nat@width\fi}
\def\maxheight{\ifdim\Gin@nat@height>\textheight\textheight\else\Gin@nat@height\fi}
\makeatother
% Scale images if necessary, so that they will not overflow the page
% margins by default, and it is still possible to overwrite the defaults
% using explicit options in \includegraphics[width, height, ...]{}
\setkeys{Gin}{width=\maxwidth,height=\maxheight,keepaspectratio}
% Set default figure placement to htbp
\makeatletter
\def\fps@figure{htbp}
\makeatother
\setlength{\emergencystretch}{3em} % prevent overfull lines
\providecommand{\tightlist}{%
  \setlength{\itemsep}{0pt}\setlength{\parskip}{0pt}}
\setcounter{secnumdepth}{-\maxdimen} % remove section numbering
% Make \paragraph and \subparagraph free-standing
\ifx\paragraph\undefined\else
  \let\oldparagraph\paragraph
  \renewcommand{\paragraph}[1]{\oldparagraph{#1}\mbox{}}
\fi
\ifx\subparagraph\undefined\else
  \let\oldsubparagraph\subparagraph
  \renewcommand{\subparagraph}[1]{\oldsubparagraph{#1}\mbox{}}
\fi
\newlength{\cslhangindent}
\setlength{\cslhangindent}{1.5em}
\newlength{\csllabelwidth}
\setlength{\csllabelwidth}{3em}
\newlength{\cslentryspacingunit} % times entry-spacing
\setlength{\cslentryspacingunit}{\parskip}
\newenvironment{CSLReferences}[2] % #1 hanging-ident, #2 entry spacing
 {% don't indent paragraphs
  \setlength{\parindent}{0pt}
  % turn on hanging indent if param 1 is 1
  \ifodd #1
  \let\oldpar\par
  \def\par{\hangindent=\cslhangindent\oldpar}
  \fi
  % set entry spacing
  \setlength{\parskip}{#2\cslentryspacingunit}
 }%
 {}
\usepackage{calc}
\newcommand{\CSLBlock}[1]{#1\hfill\break}
\newcommand{\CSLLeftMargin}[1]{\parbox[t]{\csllabelwidth}{#1}}
\newcommand{\CSLRightInline}[1]{\parbox[t]{\linewidth - \csllabelwidth}{#1}\break}
\newcommand{\CSLIndent}[1]{\hspace{\cslhangindent}#1}
% Manuscript styling
\usepackage{upgreek}
\captionsetup{font=singlespacing,justification=justified}

% Table formatting
\usepackage{longtable}
\usepackage{lscape}
% \usepackage[counterclockwise]{rotating}   % Landscape page setup for large tables
\usepackage{multirow}		% Table styling
\usepackage{tabularx}		% Control Column width
\usepackage[flushleft]{threeparttable}	% Allows for three part tables with a specified notes section
\usepackage{threeparttablex}            % Lets threeparttable work with longtable

% Create new environments so endfloat can handle them
% \newenvironment{ltable}
%   {\begin{landscape}\centering\begin{threeparttable}}
%   {\end{threeparttable}\end{landscape}}
\newenvironment{lltable}{\begin{landscape}\centering\begin{ThreePartTable}}{\end{ThreePartTable}\end{landscape}}

% Enables adjusting longtable caption width to table width
% Solution found at http://golatex.de/longtable-mit-caption-so-breit-wie-die-tabelle-t15767.html
\makeatletter
\newcommand\LastLTentrywidth{1em}
\newlength\longtablewidth
\setlength{\longtablewidth}{1in}
\newcommand{\getlongtablewidth}{\begingroup \ifcsname LT@\roman{LT@tables}\endcsname \global\longtablewidth=0pt \renewcommand{\LT@entry}[2]{\global\advance\longtablewidth by ##2\relax\gdef\LastLTentrywidth{##2}}\@nameuse{LT@\roman{LT@tables}} \fi \endgroup}

% \setlength{\parindent}{0.5in}
% \setlength{\parskip}{0pt plus 0pt minus 0pt}

% \usepackage{etoolbox}
\makeatletter
\patchcmd{\HyOrg@maketitle}
  {\section{\normalfont\normalsize\abstractname}}
  {\section*{\normalfont\normalsize\abstractname}}
  {}{\typeout{Failed to patch abstract.}}
\patchcmd{\HyOrg@maketitle}
  {\section{\protect\normalfont{\@title}}}
  {\section*{\protect\normalfont{\@title}}}
  {}{\typeout{Failed to patch title.}}
\makeatother
\shorttitle{Gaze cue understanding}
\keywords{social cognition, individual differences, gaze cues, psychometrics\newline\indent Word count: X}
\usepackage{lineno}

\linenumbers
\usepackage{csquotes}
\ifXeTeX
  % Load polyglossia as late as possible: uses bidi with RTL langages (e.g. Hebrew, Arabic)
  \usepackage{polyglossia}
  \setmainlanguage[]{english}
\else
  \usepackage[main=english]{babel}
% get rid of language-specific shorthands (see #6817):
\let\LanguageShortHands\languageshorthands
\def\languageshorthands#1{}
\fi
\ifLuaTeX
  \usepackage{selnolig}  % disable illegal ligatures
\fi


\authornote{

Add complete departmental affiliations for each author here. Each new line herein must be indented, like this line.

Enter author note here.

Correspondence concerning this article should be addressed to Julia Prein, Max Planck Institute for Evolutionary Anthropology, Deutscher Platz 6, 04103 Leipzig, Germany. E-mail: \href{mailto:julia_prein@eva.mpg.de}{\nolinkurl{julia\_prein@eva.mpg.de}}

}

\affiliation{\vspace{0.5cm}\textsuperscript{1} Department of Comparative Cultural Psychology, Max Planck Institute for Evolutionary Anthropology}

\abstract{
There must be an abstract of no more than 250 words.
One or two sentences providing a \textbf{basic introduction} to the field, comprehensible to a scientist in any discipline.

Two to three sentences of \textbf{more detailed background}, comprehensible to scientists in related disciplines.

One sentence clearly stating the \textbf{general problem} being addressed by this particular study.

One sentence summarizing the main result (with the words ``\textbf{here we show}'' or their equivalent).

Two or three sentences explaining what the \textbf{main result} reveals in direct comparison to what was thought to be the case previously, or how the main result adds to previous knowledge.

One or two sentences to put the results into a more \textbf{general context}.

Two or three sentences to provide a \textbf{broader perspective}, readily comprehensible to a scientist in any discipline.
}



\begin{document}
\maketitle

Idea for an opener :)

Developmental psychology is facing a dilemma: many research questions are questions about individual differences, yet, there is a lack of tasks to reliably measure these individual differences. For example \ldots{} .

\hypertarget{methods}{%
\section{Methods}\label{methods}}

We report how we determined our sample size, all data exclusions (if any), all manipulations, and all measures in the study.

\hypertarget{participants}{%
\subsection{Participants}\label{participants}}

\hypertarget{material}{%
\subsection{Material}\label{material}}

Material is presented as an interactive web-app that is accessible for computers and tablets and runs on any web browser.
The code is open-source (\url{https://github.com/ccp-eva/gafo-demo})
and a live demo version can be found under: \url{https://ccp-odc.eva.mpg.de/gafo-demo/}.

The web-app generates two files:
(1) a text file (JSON) containing click responses,
and (2) a video file (webm) of the participant's webcam recording.
Data gets automatically collected and safely stored on local servers located in Leipzig, Germany.

The web-app was programmed in Vanilla JavaScript, HTML and CSS.
For stimulus presentation, a scalable vector graphic (SVG) composition was parsed.
This way, the composition scales according to the user's viewport without loss of quality,
while keepiong the aspect ratio and relative object positions.
Furthermore, SVGs allow us to precisely control and calculate the size and position of all
composite parts of the scene (e.g., pupil of the agent).

The GreenSock Animation Platform (GSAP; TODO insert citation) library was used
to animate the movement of single SVG elements.

Study instructions were pre-recorded and, therefore, constant across participants.
No interaction with the experimenter was necessary and children needed minimal assistance from their caregivers.

\hypertarget{procedure-from-pre-reg}{%
\subsection{Procedure (from pre-reg)}\label{procedure-from-pre-reg}}

An animal character (i.e., agent; sheep, monkey, or pig) is placed centrally in a window.
A balloon (i.e., target; blue, green, yellow, or red) is located in front of them.
The target then falls to the ground.
The agent's gaze follows the movement of the target,
that is, the pupils of the agent move in a way that their center aligns with the center of the target.
Participants are then asked to locate the target's position on the screen.

Visual access to the target's true location is manipulated by a hedge.
Participants either have full, partial, or no visual access to the true target location.
When partial or no information about the target location is accessible,
participants are expected to use the agent's gaze as a cue.

There are two different versions of partial and no visual access trials.
For families that use a tablet with touchscreen,
the target lands behind a hedge (i.e., hedge version).
Children are then asked to click directly on the hedge to indicate where the target is.
For families that use a computer without touchscreen, the balloon falls into a box.
Children are then asked to point to the box containing the target.
Caregivers then act as the ``digital finger'' of their children and click on the indicated box.

The dependent variable depends on the study version:
For the hedge version, our dependent variable is continuous.
Here, the dependent variable is imprecision,
which is defined as the absolute difference between the true x coordinate of the target
and the x coordinate of the participant's click on the screen.
For the box version, we use our categorical outcome (i.e., which box was clicked)
to calculate the proportion of correct responses.

There are three different trial types,
depending on the visual access to the target flight and end location.
In touch training trials, participants have full visual access to the target flight
and the target's end location.
In familiarization trials, participants have visual access to the target flight
but not to the target's end location.
In the hedge version, the target's end location is covered by a hedge.
In the box version, the target flies into a box and is therefore not visible anymore.
In test trials, participants have no visual access to the target flight
nor the end location.
In both versions, the target flight is covered by a hedge.
In the hedge version, the hedge then shrinks to cover the target's end location.
In the box version, the hedge shrinks completely.
The boxes then hide the target's end location.

Depending on the participant's device,
participants have to indicate their estimated target location directly on the hedge
(i.e., hedge version) or in one of five boxes (i.e., box version).

Participants receive 19 trials with one touch training, two familiarization trials,
and 16 test trials.
The first trial of each type comprises a voice-over description of the presented events.
Touch and familiarization trials, as well as voice-over trials, will not be included in the analysis.
Therefore, we will conduct our analysis with 15 test trials.

Target locations are generated as follows.
For the hedge version, the full width of the screen is divided into ten bins.
Exact coordinates within each bin are randomly generated.
For the box version, the target randomly lands in one out of five boxes.
Each bin/box occurs equally often.
The same bin/box can occur only twice in a row.

\hypertarget{data-analysis}{%
\subsection{Data analysis}\label{data-analysis}}

We used R {[}Version 4.1.2; R Core Team (2021){]} and the R-package \emph{papaja} {[}Version 0.1.0.9997; Aust and Barth (2020){]} for all our analyses.

\hypertarget{results}{%
\section{Results}\label{results}}

\hypertarget{discussion}{%
\section{Discussion}\label{discussion}}

\hypertarget{declarations}{%
\section{Declarations}\label{declarations}}

\hypertarget{open-practices-statement}{%
\subsection{Open practices statement}\label{open-practices-statement}}

The web application (\url{https://ccp-odc.eva.mpg.de/gafo-demo/}) described here is open source (\url{https://github.com/ccp-eva/gafo-demo}).
The datasets generated during and/or analysed during the current study are available
in the {[}gazecues-methods{]} repository, (\url{https://github.com/jprein/gazecues-methods}).
All experiments were preregistered (\url{https://osf.io/zjhsc/}).

\hypertarget{funding}{%
\subsection{Funding}\label{funding}}

This study was funded by the Max Planck Society for the Advancement of Science, a noncommercial, publicly financed scientific organization (no grant number).
We thank all the children and parents who participated in the study.

\hypertarget{conflicts-of-interest}{%
\subsection{Conflicts of interest}\label{conflicts-of-interest}}

The authors declare that they have no conflict of interest.

\hypertarget{ethics-approval}{%
\subsection{Ethics approval}\label{ethics-approval}}

\hypertarget{consent-to-participate}{%
\subsection{Consent to participate}\label{consent-to-participate}}

Informed consent was obtained from all individual participants included in the study or their legal guardians.

\hypertarget{consent-for-publication}{%
\subsection{Consent for publication}\label{consent-for-publication}}

\hypertarget{open-access}{%
\subsection{Open access}\label{open-access}}

\hypertarget{authors-contributions}{%
\subsection{Authors' contributions}\label{authors-contributions}}

optional: please review the submission guidelines from the
journal whether statements are mandatory

\newpage

\hypertarget{references}{%
\section{References}\label{references}}

\begingroup
\setlength{\parindent}{-0.5in}
\setlength{\leftskip}{0.5in}

\hypertarget{refs}{}
\begin{CSLReferences}{1}{0}
\leavevmode\vadjust pre{\hypertarget{ref-R-papaja}{}}%
Aust, F., \& Barth, M. (2020). \emph{{papaja}: {Create} {APA} manuscripts with {R Markdown}}. Retrieved from \url{https://github.com/crsh/papaja}

\leavevmode\vadjust pre{\hypertarget{ref-R-base}{}}%
R Core Team. (2021). \emph{R: A language and environment for statistical computing}. Vienna, Austria: R Foundation for Statistical Computing. Retrieved from \url{https://www.R-project.org/}

\end{CSLReferences}

\endgroup


\end{document}
